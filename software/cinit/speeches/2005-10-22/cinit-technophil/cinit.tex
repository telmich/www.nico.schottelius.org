% Version: 0.5
%\documentclass[pdf,darkblue,draft,slideColor,colorBG,total]{prosper}
%\documentclass[pdf,autumn,draft,slideColor,colorBG,total]{prosper}
\documentclass[pdf,autumn,slideColor,colorBG,total]{prosper}
%\documentclass[pdf,azure,draft,slideColor,colorBG,total]{prosper}
\usepackage[latin1]{inputenc}
\usepackage{german}
%------------------------------------------------------------------------------
% prework
\title{cinit}
\subtitle{Konzept des Init-systems}
\author{Nico Schottelius}
\email{nico-linux-cinit (BEI) schottelius.org}
\institution{}
\date{2005-10-22}
%------------------------------------------------------------------------------
\begin{document}
\maketitle
%------------------------------------------------------------------------------
\begin{slide}{Init-Systeme}
\begin{enumerate}
\item Wo?
\item Wann?
\item Wie?
\end{enumerate}
\end{slide}
%------------------------------------------------------------------------------
\begin{slide}{Sys-V-Init}
\begin{itemize}
\item Standard bei den meisten Linux-Distributionen
\item Konfiguriert durch \textit{/etc/inittab}
\item Skripte mit "`start"' und "`stop"' als Argumente
\item Verschiedene Runlevel, Verzeichnisse \textit{/etc/rcX.d} (rcS.d)
\end{itemize}
\end{slide}
%------------------------------------------------------------------------------
\begin{slide}{Sys-V-Init: Probleme}
\begin{itemize}
\item Skripte
\begin{itemize}
   \item Interpreter
   \item zus\"atzliche Logik (start/stop), unn\"otige Logik
   \item generische Skripte
   \item zwanghafte Ausf\"uhrung, \"Uberpr\"ufung beim Start
   \textit{/etc/default/*}
   \item /bin/bash statt kleinere Shell
\end{itemize}
\item Pseudo-Abh\"angigkeiten (durch Benennung, sehr fehleranf\"allig)
\item Sequentieller Ablauf (S\textit{Beginn} bis S\textit{Ende})
\item Einf\"ugen neuer Dienste (\textit{S42zservice} und \textit{S43aservice})
\end{itemize}
\end{slide}
%------------------------------------------------------------------------------
\begin{slide}{Konzept cinit}
\begin{itemize}
\item Service orientiert - keine Skripte
\item weiche ("`wants"') und harte ("`needs"') Abh\"angigkeiten
\item schneller Start durch parallele Ausf\"uhrung
\item Unterst\"utzung von Profilen
\item 
\end{itemize}
\end{slide}
%------------------------------------------------------------------------------
\begin{slide}{Was ist ein Service?}
\begin{itemize}
\item ein Verzeichnis
\item "`on"' und "`off"' zum Starten und Anhalten
\item "`on.params"' und "`off.params"' Parameter (\\n unterteilt)
\item "`on.env"' und "`off.env"' Umgebungsvariablen  (\\n unterteilt)
\item Abh\"angigkeiten ("`wants"' und "`need"')
\end{itemize}
\end{slide}
%------------------------------------------------------------------------------
\begin{slide}{Abh\"angigkeiten}
\begin{itemize}
\item symbolische Links unterhalb von wants und needs auf andere Services
\end{itemize}
\begin{verbatim}
/etc/cinit/getty/2/needs:
 hostname -> ../../../network/hostname/
 root -> ../../../mount/root/
\end{verbatim}
\end{slide}
%------------------------------------------------------------------------------
\begin{slide}{Profile}
\begin{itemize}
\item cinit als Argument "`cprofile:\textit{Profilname}"'
\item ein eigener Service (z.b. \textit{profile/dos})
\item kann auch original Startpunkt beinhalten (Abh\"angigkeit)
\item erm\"oglicht verschiedene Szenarien abzubilden
\item vermeidet dynamische und unn\"otige Konfigurationsentscheidungen beim Starten
\end{itemize}
\end{slide}
%------------------------------------------------------------------------------
\begin{slide}{cinit: Installation}
\begin{itemize}
\item paralell zu bestehendem Init
\item vom Source, Debian Paket oder gentoo emerge
\item \verb=http://linux.schottelius.org/cinit/=
\end{itemize}
\end{slide}
%------------------------------------------------------------------------------
\begin{slide}{cinit: Konfiguration}
\begin{itemize}
\item Jetzt, interaktiv
\item Und Fragen?
\item Eventuell: Essen?
\end{itemize}
\end{slide}
%------------------------------------------------------------------------------
\end{document}
